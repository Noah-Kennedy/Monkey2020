Recently, Monkey communication has become a hot topic in scientific literature.
The communication of biological monkeys has seen significant study in the literature,
and continues to shed light on human communication.
Unfortunately, few researchers have seen fit to examine the communications habits of digital
monkeys.
Current literature\cite{sub_verif_plan} shows that digital monkeys generally communicate using
TCP and HTTP, but the exact bandwidth utilization and form of the HTTP protocol used for monkey
communication has not been determined previously.
Were this to be known, researchers could use this information to make further discoveries
regarding the communication habits and social structures of digital monkeys.
Fortunately, due to the recent discovery of the Wireshark computer program, we have been able
to directly record and observe the communications of digital monkeys at the packet level.
In this paper, we will establish the bandwidth utilization of digital monkeys and the specifics
of the communications protocols they seek to communicate using a combination of conventional
software API unit testing and Wireshark packet captures.