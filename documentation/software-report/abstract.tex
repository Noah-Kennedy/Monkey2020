Digital monkeys are a recently discovered species of monkey which have recently become the
subject of significant academic study due to their significant potential for use in
extraterrestrial mining operations.
Digital monkeys, like biological monkeys, can be taught to perform many advanced tasks and are
capable of complex problem solving.
In particular, they seem to exhibit advanced pathfinding and visual cognition capabilities, as well
as a sophisticated communications protocol.
Unfortunately, few researchers have previously seen fit to study in depth the mental capacities of
digital monkeys.
Further study into the cognitive behavior of digital monkeys, although woefully underfunded,
has already lead to promising leads in fields such as extraterrestrial mining, ritual snackrifice,
the safe consumption of alcohol in nuclear fallout zones\cite{teapot}, and many additional fields.
It is for this reason that we endeavor to expand the field of human knowledge in this domain.
Previous work by this group of researchers has led to the development of a basic model of digital
monkey cognition\cite{sub_verif_plan}, however this data did not leverage significant experimental
data.
We seek to expand on that work here by leveraging data from several recent experiments to
construct a more complete picture of monkey cognition.
In this paper, we outline a novel, in-depth model of the the mental capacities of digital monkeys
across three cognitive ares: pathfinding, visual cognition, and communication, using data gained
from an ambitious set of real-world experiments we developed and performed.