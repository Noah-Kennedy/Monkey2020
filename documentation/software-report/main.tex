\documentclass[11pt]{article}
\marginparwidth 0pt
\oddsidemargin 0pt
\evensidemargin 0pt
\marginparsep 0pt

\topmargin 0pt
\textwidth 6.5 in
\textheight 8 in

\usepackage{graphicx}
\usepackage{amssymb}
\usepackage{epstopdf}
\usepackage{amsmath}
\usepackage{color}
\newcommand{\hilight}[1]{\colorbox{yellow}{#1}}

\usepackage{bm}
%\usepackage{algorithmic}
\usepackage{url}
\usepackage[
    backend=biber,
    citestyle=ieee,
]{biblatex}
\graphicspath{ {Figures/} }
\DeclareGraphicsRule{.tif}{png}{.png}{`convert #1 `dirname #1`/`basename #1 .tif`.png}

\addbibresource{references.bib}

\title{
    OO OO AA AA: A Study of Monkey Communications
}
\author{
    Noah M. Kennedy \\[3pt]
    {\small Department of Electrical Engineering and Computer Science} \\
    {\small Milwaukee School of Engineering} \\
    {\small Milwaukee, WI, USA} \\
    {\small Email: {\tt noah.kennedy.professional@gmail.com}} \\[12pt]
}

\date{}

\usepackage[parfill]{parskip}

\begin{document}
    \maketitle

    \abstract
    Recently, Monkey communication has become a hot topic in scientific literature.
    The communication of biological monkeys has seen significant study in the literature,
    and continues to shed light on human communication.
    Unfortunately, few researchers have seen fit to examine the communications habits of digital
    monkeys.
    Current literature\cite{sub_verif_plan} shows that digital monkeys generally communicate using
    TCP and HTTP, but the exact bandwidth utilization and form of the HTTP protocol used for monkey
    communication has not been determined previously.
    Were this to be known, researchers could use this information to make further discoveries
    regarding the communication habits and social structures of digital monkeys.
    Fortunately, due to the recent discovery of the Wireshark computer program, we have been able
    to directly record and observe the communications of digital monkeys at the packet level.
    In this paper, we will establish the bandwidth utilization of digital monkeys and the specifics
    of the communications protocols they seek to communicate using a combination of conventional
    software API unit testing and Wireshark packet captures.

    \newpage

    \tableofcontents

    \newpage


    \section{Introduction} \label{sec:intro}
    Digital monkeys are a recently discovered species of monkey who have recently become the
    subject of significant academic study due to their significant potential for use in
    extraterrestrial mining operations.
    Digital monkeys, like biological monkeys, can be taught to perform many advanced tasks.
    They have exceptional pathfinding skills.


    \section{Monkey Communications API}\label{sec:api}

    \section{Monkey Communications Bandwidth Utilization}\label{sec:bandwidth}

    \section*{Appendix}\label{sec:appendix}

    \pagebreak

    \printbibliography
\end{document}
